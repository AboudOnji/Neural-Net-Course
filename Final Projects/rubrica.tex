\documentclass[12pt, letterpaper, oneside]{book}

\usepackage{enumitem}
\usepackage{csquotes}
% --- PAQUETES ESENCIALES ---
\usepackage[utf8]{inputenc} % Codificación de entrada UTF-8
\usepackage[T1]{fontenc}    % Codificación de fuentes moderna
\usepackage[spanish,es-tabla]{babel} % Soporte para español, incluyendo tablas
\usepackage{lmodern}        % Fuente Latin Modern para mejor tipografía

% --- PAQUETES PARA TABLAS Y DISEÑO ---
\usepackage{longtable} % Para tablas que ocupan varias páginas
\usepackage{array}     % Para tipos de columna avanzados (como p{width}, m{width})
\usepackage{booktabs}  % Para líneas de tabla de calidad profesional (toprule, midrule, bottomrule)
\usepackage{ragged2e}  % Para mejor justificación en celdas (ej. \RaggedRight)
\usepackage[table]{xcolor} % Para colores en tablas
\usepackage{graphicx}  % Para rotar texto en celdas

% --- PAQUETES GENERALES (Mantenidos para consistencia de estilo) ---
\usepackage{amsmath}
\usepackage{amsfonts}
\usepackage{amssymb}
\usepackage{xurl}
\usepackage{hyperref}
\hypersetup{
    colorlinks=true,
    linkcolor=blue,
    filecolor=blue,
    urlcolor=blue,
    citecolor=blue,
}
\usepackage[numbers,square,sort&compress]{natbib}
\bibliographystyle{apalike}
\usepackage{geometry}
\geometry{left=3cm, right=2.5cm, top=2.5cm, bottom=2.5cm}
\usepackage{fancyhdr}
\usepackage{setspace}
\onehalfspacing
\usepackage{emptypage}
\usepackage{caption}
\captionsetup{labelfont=bf, textfont=it, skip=10pt}

% --- CONFIGURACIÓN DE ENCABEZADOS Y PIES DE PÁGINA ---
\pagestyle{fancy}
\fancyhf{}
\fancyhead[LE,RO]{\slshape\nouppercase{\leftmark}}
\fancyhead[RE,LO]{Rúbricas de Evaluación - Redes Neurales y SVM}
\fancyfoot[CE,CO]{\thepage}
\renewcommand{\headrulewidth}{0.4pt}
\renewcommand{\footrulewidth}{0pt}

% --- COLORES Y ESTILOS PARA LA RÚBRICA ---
\definecolor{TableHeader}{HTML}{EAEAEA} % Un gris claro para el encabezado de la tabla
\newcolumntype{P}[1]{>{\RaggedRight\arraybackslash}p{#1}} % Columna con justificación a la izquierda

% --- INFORMACIÓN DEL DOCUMENTO ---
\title{Redes Neurales y Support Vector Machine \\ Rúbrica de Evaluación para Proyecto Final}
\author{\textcopyright Prof. Dr. Aboud BARSEKH-ONJI\\ \url{aboud.barsekh@anahuac.mx} \\ \url{https://orcid.org/0009-0004-5440-8092}} 
\date{\today}

% --- INICIO DEL DOCUMENTO ---
\begin{document}

\maketitle
\thispagestyle{empty}
\clearpage



\chapter*{Rúbrica de Proyecto}
\label{sec:rubrica2}
\begin{longtable}{P{3.5cm} P{1.5cm} P{9cm}}
\caption{Rúbrica de evaluación para el Proyecto de Redes Neurales y Support Vector Machine} \label{tab:rubrica2} \\
\toprule
\rowcolor{TableHeader}
\textbf{Criterio de Evaluación} & \textbf{Ponderación} & \textbf{Niveles de Desempeño} \\
\midrule
\endfirsthead
\multicolumn{3}{c}%
{{\bfseries \tablename\ \thetable{} -- continuación}} \\
\toprule
\rowcolor{TableHeader}
\textbf{Criterio de Evaluación} & \textbf{Ponderación} & \textbf{Niveles de Desempeño} \\
\midrule
\endhead
\bottomrule
\endfoot
\bottomrule
\endlastfoot

% --- Criterios de Evaluación ---
\textbf{Estructura y Formato del Paper} & 10 Puntos &
\begin{itemize}[noitemsep, topsep=0pt, partopsep=0pt, leftmargin=*]
    \item \textbf{Excelente (9-10):} Sigue impecablemente todas las secciones, formatos y lineamientos solicitados. El documento es profesional y fácil de leer.
    \item \textbf{Bueno (7-8):} Cumple con la mayoría de los lineamientos, pero puede tener errores menores en formato o estructura.
    \item \textbf{Regular (5-6):} Faltan secciones importantes o el formato es inconsistente.
    \item \textbf{Deficiente (0-4):} No sigue la estructura de un paper científico.
\end{itemize} \\
\addlinespace

\textbf{Metodología: Etapas de Modelado, Validación y Aprendizaje Automático} & 50 Puntos &
\begin{itemize}[noitemsep, topsep=0pt, partopsep=0pt, leftmargin=*]
    \item \textbf{Excelente (45-50):} Todas las etapas de la metodología están descritas de forma explícita, detallada y rigurosa. La lógica del modelo, su validación y los experimentos son impecables.
    \item \textbf{Bueno (38-44):} Se describen la mayoría de las etapas, pero alguna podría tener falta de detalle o rigor.
    \item \textbf{Regular (30-37):} Faltan etapas clave (ej. validación) o la descripción es superficial. El modelo puede tener fallos lógicos.
    \item \textbf{Deficiente (0-29):} No se sigue el proceso metodológico. La metodología es inexistente o muy pobre.
\end{itemize} \\
\addlinespace

\textbf{Análisis y Comparación de Resultados} & 25 Puntos &
\begin{itemize}[noitemsep, topsep=0pt, partopsep=0pt, leftmargin=*]
    \item \textbf{Excelente (21-25):} El análisis estadístico es robusto y adecuado. La interpretación de los resultados y la comparación de escenarios son profundas y aportan valor.
    \item \textbf{Bueno (17-20):} El análisis es correcto, pero la interpretación o la discusión podrían ser más profundas.
    \item \textbf{Regular (12-16):} El análisis es muy básico o presenta errores. La discusión es superficial.
    \item \textbf{Deficiente (0-11):} No hay análisis de resultados o es incorrecto.
\end{itemize} \\
\addlinespace

\textbf{Conclusiones y Recomendaciones} & 10 Puntos &
\begin{itemize}[noitemsep, topsep=0pt, partopsep=0pt, leftmargin=*]
    \item \textbf{Excelente (9-10):} Las conclusiones se derivan directamente de los resultados. Las recomendaciones son prácticas, viables y bien fundamentadas.
    \item \textbf{Bueno (7-8):} Las conclusiones son correctas, pero las recomendaciones podrían ser más específicas o justificadas.
    \item \textbf{Regular (5-6):} Las conclusiones no se conectan claramente con los resultados.
    \item \textbf{Deficiente (0-4):} No hay conclusiones o no son pertinentes.
\end{itemize} \\
\addlinespace

\textbf{Referencias y Redacción} & 5 Puntos &
\begin{itemize}[noitemsep, topsep=0pt, partopsep=0pt, leftmargin=*]
    \item \textbf{Excelente (5):} No hay errores de redacción u ortografía. Las referencias siguen el formato APA 7 o IEEE.
    \item \textbf{Bueno (4):} Errores menores de redacción o en el formato de referencias.
    \item \textbf{Regular (3):} Errores que dificultan la lectura.
    \item \textbf{Deficiente (0-2):} Calidad de escritura muy baja.
\end{itemize} \\
\midrule
\textbf{TOTAL} & \textbf{100 Puntos} & \\

\end{longtable}

\end{document}
