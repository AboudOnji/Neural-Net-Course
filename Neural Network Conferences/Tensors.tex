\documentclass[aspectratio=169,xcolor=dvipsnames]{beamer}
\usetheme{Berlin}

\usepackage[english]{babel}
\usepackage{hyperref}
\usepackage{graphicx}
\usepackage{booktabs}
\usepackage{amsmath}
\usepackage{lettrine}
\setbeamertemplate{caption}[numbered]
\usepackage[dvipsnames,svgnames,x11names]{xcolor}
\usepackage{xurl}
\usepackage{algorithm}
\usepackage{algorithmicx}
\usepackage{algpseudocode}
\usepackage{adjustbox}
\hypersetup{
    colorlinks=true,
    linkcolor=cyan,
    filecolor=blue,
    urlcolor=blue,
    citecolor=blue,
}

%----------------------------------------------------------------------------------------
%	CODE LISTINGS SETTINGS
%----------------------------------------------------------------------------------------
\usepackage{listings}
\usepackage{xcolor}

\definecolor{codegreen}{rgb}{0,0.6,0}
\definecolor{codegray}{rgb}{0.5,0.5,0.5}
\definecolor{codepurple}{rgb}{0.58,0,0.82}
\definecolor{backcolour}{rgb}{0.97,0.97,0.99}

\lstdefinestyle{MATLABStyle}{
  language=Matlab,
  basicstyle=\ttfamily\scriptsize, % Reducido ligeramente para caber en diapositivas
  keywordstyle=\color{blue}\bfseries,
  commentstyle=\color{codegreen},
  stringstyle=\color{violet},
  numberstyle=\tiny\color{gray},
  breakatwhitespace=false,
  breaklines=true,
  captionpos=b,
  keepspaces=true,
  numbers=left,
  numbersep=5pt,
  showspaces=false,
  showstringspaces=false,
  showtabs=false,
  tabsize=2,
  frame=lines,
  framerule=0.4pt,
  backgroundcolor=\color{backcolour}
}
\lstset{style=MATLABStyle}

%----------------------------------------------------------------------------------------
%	TITLE PAGE INFO
%----------------------------------------------------------------------------------------

\title{Data Structures in Deep Learning}
\subtitle{Understanding Tensors: Dimensions C, B, T, S}

\author{Prof. Dr. BARSEKH-ONJI Aboud}

\institute
{
    Facultad de Ingeniería \\
    Universidad Anáhuac México
}
\date{\today}

%----------------------------------------------------------------------------------------
%	PRESENTATION SLIDES
%----------------------------------------------------------------------------------------

\AtBeginSection[]
{
  \begin{frame}{Agenda}
    \tableofcontents[currentsection]
  \end{frame}
}

\begin{document}

\begin{frame}
    \titlepage
\end{frame}

%------------------------------------------------
\section{Fundamentals of Tensors}
%------------------------------------------------

\begin{frame}{Beyond the Matrix}
    \begin{block}{What is a Tensor?}
    In computational intelligence, a tensor is a generalization of scalars, vectors, and matrices to higher dimensions. It is the fundamental data structure for Deep Learning frameworks (MATLAB, PyTorch, TensorFlow).
    \end{block}

    \begin{table}[]
    \centering
    \begin{tabular}{@{}lll@{}}
    \toprule
    \textbf{Rank} & \textbf{Name} & \textbf{MATLAB Representation} \\ \midrule
    0 & Scalar & \texttt{Single value (1x1)} \\
    1 & Vector & \texttt{Array (1xN)} \\
    2 & Matrix & \texttt{Grayscale Image (MxN)} \\
    3+ & Tensor & \texttt{Color Image / Batch Data} \\ \bottomrule
    \end{tabular}
    \end{table}
\end{frame}

\begin{frame}{Why Dimensions Matter?}
    \begin{alertblock}{The "Dimension Mismatch" Problem}
    Most compilation errors in Deep Network Designer arise from incompatible tensor shapes. Understanding the labels \textbf{C, B, T, S} is crucial for:
    \end{alertblock}
    
    \begin{itemize}
        \item Connecting layers correctly (e.g., LSTM to Dense Layer).
        \item Shaping input data from sensors.
        \item Debugging memory errors in GPUs.
    \end{itemize}
\end{frame}

%------------------------------------------------
\section{The Core Dimensions}
%------------------------------------------------

\begin{frame}{The C-Dimension (Channel)}
            \begin{block}{C: Channel / Features}
            Represents the "depth" of information at a single point. This dimension is \textbf{fixed} by the network architecture.
            \end{block}
            \begin{itemize}
                \item \textbf{In Time Series:} The number of sensors or variables.
                \begin{itemize}
                    \item Accelerometer (x, y, z) $\rightarrow C=3$.
                    \item Medical Data (HR, Temp, SPO2) $\rightarrow C=3$.
                \end{itemize}
                \item \textbf{In Vision:} The color planes (RGB $\rightarrow C=3$).
            \end{itemize}
            \vspace{1cm}
            \centering
            % Placeholder for illustration

\end{frame}

\begin{frame}{The B-Dimension (Batch)}
    \begin{block}{B: Batch / Observation}
    The dimension of \textbf{computational parallelism}. It defines how many independent examples are processed simultaneously by the GPU.
    \end{block}
    
    \begin{itemize}
        \item \textbf{Batch Training:} Instead of updating weights after 1 sample, we average the gradient of $N$ samples (MiniBatchSize).
        \item \textbf{Impact:} Higher $B$ improves stability and speed but requires more VRAM.
        \item During inference (real-time use), usually $B=1$.
    \end{itemize}
\end{frame}

\begin{frame}{The T-Dimension (Time)}
    \begin{alertblock}{T: Time / Sequence}
    Exclusive to Sequence Models (RNN, LSTM, GRU). Represents the duration of the event.
    \end{alertblock}
    
    \begin{itemize}
        \item \textbf{Variable Length:} Unlike $C$, $T$ can vary. A trained LSTM can process a signal of $T=50$ and then $T=500$.
        \item \textbf{Recurrent Processing:} The network unfolds loop $T$ times:
        \begin{equation}
             h_t = \sigma(W x_t + U h_{t-1} + b)
        \end{equation}
        where $t$ iterates from $1$ to $T$.
    \end{itemize}
\end{frame}

\begin{frame}{The S-Dimension (Spatial)}
    \begin{block}{S: Spatial (H/W)}
    Represents geometric dimensions (Height and Width).
    \end{block}
            \begin{itemize}
                \item \textbf{Context:} Convolutional Neural Networks (CNNs).
                \item \textbf{Deep Network Designer:}
                \begin{itemize}
                    \item Images: $H \times W \times C \times B$ (`SSCB`).
                    \item Time Series: Spatial dimension is implicitly 1. Usually labeled just `CBT`.
                \end{itemize}
            \end{itemize}
\end{frame}

%------------------------------------------------
\section{Implementation in MATLAB}
%------------------------------------------------

\begin{frame}[fragile]{MATLAB: The dlarray Object}
    Modern MATLAB (R2023a+) uses formatted `dlarray` objects to explicitly handle these dimensions.
    
    \begin{lstlisting}[language=Matlab, caption=Creating a labeled Tensor for LSTM training]
% Example: 100 samples of 3-axis accelerometer data (5 seconds @ 50Hz)
numSamples = 100;    % Batch (B)
numSensors = 3;      % Channels (C)
timeSteps = 250;     % Time (T) -> 5s * 50Hz

% Create Random Data
rawData = randn(numSensors, numSamples, timeSteps); 

% Convert to labeled Deep Learning Array
X = dlarray(rawData, "CBT"); % Ordering: Channel, Batch, Time

disp("Tensor Dimensions: " + dims(X))
    \end{lstlisting}
\end{frame}

\begin{frame}[fragile]{Analyzing Dimensions in Network Analyzer}
    When designing a network (e.g., for `WaveformData`), check the input layer:
    
    \begin{lstlisting}[language=Matlab]
    layer = sequenceInputLayer(3, "Name", "input");
    % Here '3' fixes the C-dimension.
    \end{lstlisting}
    
    \begin{alertblock}{Validation Rules}
    \begin{enumerate}
        \item \textbf{Input Check:} Your data's $C$ must match the input layer's size exactly.
        \item \textbf{Output Check:} The classification layer must match the number of classes.
        \item $B$ and $T$ are flexible during design but consume memory during execution.
    \end{enumerate}
    \end{alertblock}
\end{frame}

%------------------------------------------------
\section{Conclusion}
%------------------------------------------------

\begin{frame}{Summary}
    \begin{itemize}
        \item \textbf{Tensors} are the fuel of Neural Networks.
        \item \textbf{C (Channels):} Structural depth (sensors/features). Fixed by design.
        \item \textbf{B (Batch):} Parallel observations. affects training speed/stochasticity.
        \item \textbf{T (Time):} Sequence length. Handles dynamic temporal patterns.
        \item \textbf{S (Spatial):} Geometric layout for vision tasks.
    \end{itemize}
    
\end{frame}

\end{document}